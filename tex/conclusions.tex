\section{Conclusions and Future Work}
\label{sec:conclusions}

This paper presents a transformation from OpenAPI to Petri nets. We have developed a tool that supports this transformation, called {\nameTool}, which is publicly available and freely accessible in our GitHub repository. Our evaluation showed that  {\nameTool} can detect successful Broken Object Level Authorization attacks (the first OWASP Top 10 2019 security risk in web applications) in web server logs with more than $95\%$ accuracy.

There are still some limitations that can be addressed in future work. First, our tool assumes the existence of a well-formed OpenAPI specification with the attributes required for analysis, like other previous works~\cite{DBLP:journals/corr/abs-2201-10833, haddad2022openapi}. Our ongoing efforts aim to somewhat relax this assumption and instead require instead source code annotations to parse and build the input model needed for the transformation. Second, logs must be centralized. This can be difficult in heavily distributed applications, but it may be feasible in practice. Similarly, in the case of multiple log files referring to the same period, a chronological sorting of the logs is required before processing. Lastly, our tool only performs offline analysis. A big step to protect the security of RESTful web services would be to integrate our tool directly into the web server, analyzing requests in real time. We envision eBPF technology as a possible solution to do this.
